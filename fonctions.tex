\usepackage{xparse}
%%%
% Fonctions pour le bon fonctionnement de la FEUS
%%%

\newcounter{annexe}
\newcounter{point}
\newcounter{proposition}[point]
\newcounter{subpoint}[point]
\newcounter{subsubpoint}[subpoint]
\newcounter{subsubsubpoint}[subsubpoint]

%%%
% Macros de postes FEUS
%%%
\NewDocumentCommand{\Admin}{s}{Administrat\IfBooleanTF{#1}{rice}{eur} Annuel\IfBooleanTF{#1}{le}{}}
\NewDocumentCommand{\prez}{s}{\IfBooleanTF{#1}{présidente}{président}}
\NewDocumentCommand{\Prez}{s}{\IfBooleanTF{#1}{Présidente}{Président}}
\NewDocumentCommand{\ractivite}{s}{\IfBooleanTF{#1}{Responsable aux activités étudiantes}{responsable aux activités étudiantes}}
\NewDocumentCommand{\ranp}{s}{\IfBooleanTF{#1}{Responsable aux affaires nationales et politiques}{responsable aux affaires nationales et politiques}}
\NewDocumentCommand{\raur}{s}{\IfBooleanTF{#1}{Responsable aux affaires universitaires et à la recherche}{responsable aux affaires universitaires et à la recherche}}
\NewDocumentCommand{\rcede}{s}{\IfBooleanTF{#1}{Responsable à la condition étudiante et aux droits des étudiants}{responsable à la condition étudiante et aux droits des étudiants}}
\NewDocumentCommand{\rddalc}{s}{\IfBooleanTF{#1}{Responsable au développement durable et aux affaires locales et communautaires}{responsable au développement durable et aux affaires locales et communautaires}}
\NewDocumentCommand{\rexec}{s}{\IfBooleanTF{#1}{Responsable à l'exécutif}{responsable à l'exécutif}}
\NewDocumentCommand{\rexterne}{s}{\IfBooleanTF{#1}{Responsable aux affaires externes}{responsable aux affaires externes}}
\NewDocumentCommand{\rrc}{s}{\IfBooleanTF{#1}{Responsable aux relations et communications}{responsable aux relations et communications}}
\NewDocumentCommand{\rsg}{s}{\IfBooleanTF{#1}{Responsable au secrétariat général}{responsable au secrétariat général}}
\NewDocumentCommand{\sg}{s}{secrétaire général\IfBooleanTF{#1}{e}{}}
\NewDocumentCommand{\SG}{s}{Secrétaire général\IfBooleanTF{#1}{e}{}}
\NewDocumentCommand{\preza}{s}{\IfBooleanTF{#1}{présidente d'assemblée}{président d'assemblée}}
\newcommand{\seca}{secrétaire d'assemblée}
\NewDocumentCommand{\vpactivite}{s}{vice-président\IfBooleanTF{#1}{e}{} aux activités étudiantes}
\NewDocumentCommand{\VPActivite}{s}{Vice-président\IfBooleanTF{#1}{e}{} aux activités étudiantes}
\NewDocumentCommand{\vpaur}{s}{vice-président\IfBooleanTF{#1}{e}{} aux affaires universitaires et à la recherche}
\NewDocumentCommand{\VPAUR}{s}{Vice-président\IfBooleanTF{#1}{e}{} aux affaires universitaires et à la recherche}
\NewDocumentCommand{\vpce}{s}{vice-président\IfBooleanTF{#1}{e}{} à la condition étudiante}
\NewDocumentCommand{\VPCE}{s}{Vice-président\IfBooleanTF{#1}{e}{} à la condition étudiante}
\NewDocumentCommand{\vpddalc}{s}{vice-président\IfBooleanTF{#1}{e}{} au développement durable,\ aux\ affaires\ locales\ et\ communautaires}
\NewDocumentCommand{\VPDDALC}{s}{Vice-président\IfBooleanTF{#1}{e}{} au développement durable,\ aux\ affaires\ locales\ et\ communautaires}
\NewDocumentCommand{\vpex}{s}{vice-président\IfBooleanTF{#1}{e}{} aux affaires externes}
\NewDocumentCommand{\VPEX}{s}{Vice-président\IfBooleanTF{#1}{e}{} aux affaires externes}
\NewDocumentCommand{\vprc}{s}{vice-président\IfBooleanTF{#1}{e}{} aux communications}
\NewDocumentCommand{\VPRC}{s}{Vice-président\IfBooleanTF{#1}{e}{} aux communications}

%%%
% Macros d'adoptions communs
%%%
\newcommand{\aam}{L'amendement est adopté à majorité.}
\newcommand{\aau}{L'amendement est adopté à l'unanimité.}
\newcommand{\aaua}{L'amendement est adopté à l'unanimité, avec abstentions.}
\newcommand{\abau}{L'amendement est battu à l'unanimité.}
\newcommand{\abm}{L'amendement est battu à majorité.}
\newcommand{\pam}{La proposition est adoptée à majorité.}
\newcommand{\pau}{La proposition est adoptée à l'unanimité.}
\newcommand{\paua}{La proposition est adoptée à l'unanimité, avec abstentions.}
\newcommand{\pbau}{La proposition est battue à l'unanimité.}
\newcommand{\pbm}{La proposition est battue à majorité.}
\newcommand{\qpau}{La question préalable passe à l'unanimité.}
\newcommand{\qpam}{La question préalable passe aux deux tiers.}
\newcommand{\saam}{Le sous-amendement est adopté à majorité.}
\newcommand{\saau}{Le sous-amendement est adopté à l'unanimité.}
\newcommand{\saaua}{Le sous-amendement est adopté à l'unanimité, avec abstentions.}
\newcommand{\sabm}{Le sous-amendement est battu à majorité.}

%%%
% Macros d'instances FEUS
%%%
\NewDocumentCommand{\ag}{s}{\IfBooleanTF{#1}{Assemblée générale}{assemblée générale}}
\NewDocumentCommand{\aga}{s}{\IfBooleanTF{#1}{Assemblée générale annuelle}{assemblée générale annuelle}}
\NewDocumentCommand{\ca}{s}{\IfBooleanTF{#1}{Conseil d'administration}{conseil d'administration}}
\NewDocumentCommand{\cac}{s}{\IfBooleanTF{#1}{Comité d'étude des affaires des campi}{comité d'étude des affaires des campi}}
\NewDocumentCommand{\cae}{s}{\IfBooleanTF{#1}{Comité d'étude des affaires externes}{comité d'étude des affaires externes}}
\NewDocumentCommand{\cm}{s}{\IfBooleanTF{#1}{Conseil des membres}{conseil des membres}}
\NewDocumentCommand{\cms}{s}{\IfBooleanTF{#1}{Conseil des membres spécial}{conseil des membres spécial}}
\NewDocumentCommand{\cs}{s}{\IfBooleanTF{#1}{Commission Spéciale}{commission spéciale}}

%%%
% Macros de texte divers
%%%
\newcommand{\appel}{Appel à la décision\newline du président}
\newcommand{\cdi}{Carrefour de l'information}
\newcommand{\cdf}{Centre de foires de Sherbrooke}
\newcommand{\dd}{développement durable}
\newcommand{\flsh}{Faculté des Lettres et Sciences Humaines}
\newcommand{\fmss}{Facuté de Médecine et des Sciences de la Santé}
\newcommand{\jrd}{Joly Riendeau \& Associé inc. CPA}
\newcommand{\mmgs}{Melançon, Marceau, Grenier et Sciortino}
\newcommand{\odj}{ordre du jour}
\newcommand{\racj}{Régie des Alcools, Courses et Jeux (RACJ)}
\newcommand{\rcgt}{Raymond Chabot Grant Thornton}
\newcommand{\RGs}{Règlements Généraux}
\newcommand{\sciences}{Faculté des Sciences}
\newcommand{\sps}{Service de Police de Sherbrooke (SPS)}
\newcommand{\sve}{Service à la Vie Étudiante (SVE)}
\newcommand{\udes}{Université de Sherbrooke}

%%%
% Macros de style
%%%
\NewDocumentCommand{\amendement}{smm}{
	\saut
	& \ifbool{useUnderline}{\uline{Amendement~:}}{Amendement~:} & \par{\textbf{#2 \IfBooleanTF{#1}{}{Amendement appuyé par #3.}}} \\
	\IfBooleanTF{#1}{\resultat{L'amendement tombe faute d'appui.}}{}
}

\NewDocumentCommand{\annexe}{mo}{
	\clearpage
	\renewcommand{\theannexe}{\theinstance.\Alph{annexe}}
	\refstepcounter{annexe}\IfNoValueTF{#2}{}{\label{anx:#2}}
	\addcontentsline{toc}{chapter}{\instance\ \ifbool{showYear}{\theannee.}{}\ifbool{showSession}{\thesession/}{}\theannexe \hspace{2em} #1}
	\vspace*{\fill}
	\begin{center}
		\huge{ Annexe~\theannexe\\
		#1}
	\end{center}
	\vspace{\fill}
	\clearpage
}

\newcommand{\appui}[1]{
	#1\ appuie.
}

\NewDocumentCommand{\caucus}{mmoo}{
	% Arguments:
	% #1 -> Requester
	% #2 -> Duration
	% #3 -> Begin time (optional)
	% #4 -> End time (optional)
	\procedure[Caucus]{#1 demande un caucus d'une durée de #2.\IfNoValueTF{#3}{}{\IfNoValueTF{#4}{}{\newline Début du caucus à #3.\newline Fin du caucus à #4.}}}
}

\newcommand{\discussion}[1]{
	& & #1 \\
}

\NewDocumentCommand{\nomination}{smmo}{
	% * -> Flags a refused nomination
	% #2 -> Proposer
	% #3 -> Nominee
	% #4 -> Additionnal flag
	\saut
	\procedure{#2 nomine #3\IfNoValueTF{#4}{}{ au poste de #4}.}
	\procedure{#3 \IfBooleanTF{#1}{refuse}{accepte} la nomination\IfNoValueTF{#4}{}{ au poste de #4}.}
}

%Point
\renewcommand{\thepoint}{\theinstance.\arabic{point}}
\NewDocumentCommand{\point}{smo}{
	\saut
	\refstepcounter{point}\IfNoValueTF{#3}{}{\label{pt:#3}}
	\instance\ \ifbool{showYear}{\theannee.}{}\ifbool{showSession}{\thesession/}{}\thepoint & \multicolumn{2}{l}{\ifbool{useUnderline}{\uline{#2}}{#2}} \\
	\addcontentsline{toc}{chapter}{\instance\ \ifbool{showYear}{\theannee.}{}\ifbool{showSession}{\thesession/}{}\thepoint \hspace{2em} #2}
	\IfBooleanTF{#1}{& & \textit{Point statutaire.} \\}{}
}

% Annotation de procedure
\NewDocumentCommand{\procedure}{om}{
% 	%%% USAGE:
% 	% \procedure[]{}
% 	% [] -> Privileged Procedure Indicator
% 	% {} -> Procedure text
	& \IfNoValueTF{#1}{}{\ifbool{useUnderline}{\uline{#1~:}}{#1~:}} & \IfNoValueTF{#1}{\textit{#2}}{\textbf{#2}} \\
}

\NewDocumentCommand{\proposition}{smm}{
% Arguments
% #1 -> * flag non appuyé
% #2 -> Proposeur et libellé
% #3 -> Appuyeur
	\refstepcounter{proposition}\saut
	\ifbool{showResolutionNo}{\multicolumn{1}{c}{(\ifbool{showInstanceNoInResolutionNo}{\instance\ \ifbool{showYear}{\theannee.}{}\ifbool{showSession}{\thesession/}{}}{}\thepoint\ifnumcomp{\value{subpoint}}{>}{0}{.\arabic{subpoint}\ifnumcomp{\value{subsubpoint}}{>}{0}{.\arabic{subsubpoint}\ifnumcomp{\value{subsubsubpoint}}{>}{0}{.\arabic{subsubsubpoint}}{}}{}}{}.\theproposition)}}{} & \ifbool{useUnderline}{\uline{Proposition~:}}{Proposition~:} & \par{\noindent\textbf{#2 \IfBooleanTF{#1}{}{Proposition appuyée par #3.}}} \\
	\IfBooleanTF{#1}{\resultat{La proposition tombe faute d'appui.}}{}
}

\newcommand{\qp}[1]{ %Question Préalable
	\saut
	& \ifbool{useUnderline}{\uline{Question Préalable~:}}{Question Préalable~:} & \par{\textbf{#1 demande la question préalable.}} \\
}

\NewDocumentCommand{\resultat}{ooom}{
% Arguments:
% #1 Pros (Optional)
% #2 Cons (Optional)
% #3 Abstentions (Optional)
% #4 Conclusion (Mandatory)
	\saut
	& \ifbool{useUnderline}{\uline{Résultat~:}}{Résultat~:} & \par{\IfNoValueTF{#1}{}{Pour~:~#1\newline}\IfNoValueTF{#2}{}{Contre~:~#2\newline}\IfNoValueTF{#3}{}{Abstention~:~#3\newline}\textbf{#4}} \\
	\saut
}

\newcommand{\saut}{
% Skips a line
	& \\
}

\NewDocumentCommand{\sousamendement}{smm}{
	\saut
	& \ifbool{useUnderline}{\uline{Sous-amendement~:}}{Sous-amendement~:} & \par{\textbf{#2 \IfBooleanTF{#1}{}{Sous-amendement appuyé par #3.}}} \\
	\IfBooleanTF{#1}{\resultat{L'amendement tombe faute d'appui.}}{}
}

% Sous point: x.y.z
\renewcommand{\thesubpoint}{\thepoint.\arabic{subpoint}}
\NewDocumentCommand{\souspoint}{smo}{
  \setcounter{proposition}{0} % Reset proposition counter
  \saut
  \hangindent=2cm
  \refstepcounter{subpoint}\IfNoValueTF{#3}{}{\label{spt:#3}}
  \IfBooleanTF{#1}{
    & \multicolumn{2}{m{12cm}}{\thesubpoint\hspace{2em}#2} \\
    & & \textit{Point statutaire.} \\
  }{
    & \multicolumn{2}{m{12cm}}{\thesubpoint\hspace{2em}#2} \\
  }
  \hangindent=0cm
  \addcontentsline{toc}{section}{\thesubpoint\hspace{2em} #2}
}

% Sous Sous Point: x.y.z.a
\renewcommand{\thesubsubpoint}{\thesubpoint.\arabic{subsubpoint}}
\NewDocumentCommand{\soussouspoint}{smo}{
  \setcounter{proposition}{0} % Reset proposition counter
  \saut
  \refstepcounter{subsubpoint}\IfNoValueTF{#3}{}{\label{sspt:#3}}
  \hangindent=2cm
  \IfBooleanTF{#1}{
    & \multicolumn{2}{m{12cm}}{\hspace{2em}\thesubsubpoint\hspace{1.5em}#2} \\
    & & \textit{Point statutaire.} \\
  }{
    & \multicolumn{2}{m{12cm}}{\hspace{2em}\thesubsubpoint\hspace{1.5em}#2} \\
  }
  \hangindent=0cm
  \addcontentsline{toc}{subsection}{\hspace{2em}\thesubsubpoint\hspace{1.5em} #2}
}

% Sous Sous Sous Point: x.y.z.a.b
\renewcommand{\thesubsubsubpoint}{\thesubsubpoint.\arabic{subsubsubpoint}}
\NewDocumentCommand{\soussoussouspoint}{smo}{
  \setcounter{proposition}{0} % Reset proposition counter
  \saut
  \refstepcounter{subsubsubpoint}\IfNoValueTF{#3}{}{\label{ssspt:#3}}
  \hangindent=2cm
  \IfBooleanTF{#1}{
    & \multicolumn{2}{m{12cm}}{\hspace{2em}\thesubsubsubpoint\hspace{1.5em}#2} \\
    & & \textit{Point statutaire.} \\
  }{
    & \multicolumn{2}{m{12cm}}{\hspace{2em}\thesubsubsubpoint\hspace{1.5em}#2} \\
  }
  \hangindent=0cm
  \addcontentsline{toc}{subsection}{\hspace{2em}\thesubsubsubpoint\hspace{1.5em} #2}
}

% Vote -> Demande de vote
\NewDocumentCommand{\vote}{m}{
	\procedure{#1 demande le vote.}
}

%%%
% Tableau de présences/statistiques
%%%
\newcommand{\presence}[2]{
#1 & #2 \\
}

\newcommand{\titrepresence}[1]{
\multicolumn{2}{@{}l}{\centering{\textbf{\uppercase{#1}}}}\\
}

%%% 
% Statistical counters
%%%
\newcounter{interventions}
\newcounter{interventionsh}
\newcounter{interventionsf}
\newcounter{offsethommes}
\newcounter{offsetfemmes}

% Step the intervention counters
\newcommand{\interventionhomme}{\stepcounter{interventions}\stepcounter{interventionsh}}
\newcommand{\interventionfemme}{\stepcounter{interventions}\stepcounter{interventionsf}}

\newcommand{\fixInterventions}{
	% fixes the disparities from presence table
	\addtocounter{interventionsh}{-\theoffsethommes}
	\addtocounter{interventionsf}{-\theoffsetfemmes}
}

\newcommand{\resetAssoCounters}{
	% reset all association counters to 0 after the members table
	% Useful to drop in praesidium offsets
	\setcounter{offsethommes}{\theinterventionsh}
	\setcounter{offsetfemmes}{\theinterventionsf}
	\setcounter{interventions}{0}
	\setcounter{interventionsf}{0}
	\setcounter{interventionsh}{0}
	\setcounter{ADEEP}{0}
	\setcounter{AGED}{0}
	\setcounter{AGEEFEUS}{0}
	\setcounter{AGEEMUS}{0}
	\setcounter{AGEEP}{0}
	\setcounter{AGEESIUS}{0}
	\setcounter{AGEFLESH}{0}
	\setcounter{AGEG}{0}
	\setcounter{AGER}{0}
	\setcounter{AGES}{0}
	\setcounter{REGS}{0}
	\setcounter{FEUS}{0}
}

% Statistics table functions
\newcommand{\titrestats}[1]{
	\hline\multicolumn{2}{||c||}{\textbf{\uppercase{#1}}} \\
}
\newcommand{\statperso}[2]{
#1 & #2 \\
}
\newcommand{\statsasso}[2]{
	\hline\hline
	#1 & #2 \\\hline
}
\newcommand{\statskip}{
\hline
\multicolumn{2}{l}{}\\
}

%%%
% Technical functions
%%%

% Replace "Table des matieres" with "Ordre du jour"
\addto\captionsfrench{
	\renewcommand{\contentsname}{\uppercase{Ordre du Jour}}
}

% Officers signatures box
\newcommand{\signatures}[2]{
	\begin{tabularx}{\textwidth}{XXX}
	& & \\
	& & \\
	& & \\\cline{1-1}\cline{3-3}
	#1 & & #2\\
	Présidence d'assemblée & & Secrétariat d'assemblée\\
	\end{tabularx}

}

% Enables BEST OF functionnality
\newbool{bestof}
%\booltrue{bestof} % For Best of version
\boolfalse{bestof} % For sane version

% Document Color Specs
% \color{white}
% \pagecolor{black}

%%%
% REGEXES
%%%

% Reference REGEXes for faster editing:

% Arrivals:
% Shorthand Notation:
% +\macro time
% Selectors:
% /% \+(\w*) (\d\d?h\d?\d?)\.?/
% Replace:
% /\\procedure{Arrivée de \\\1\\ à \2.}/

% Capture all discussion entries and format them right (Use after everything else is set)
% Selector:
% /^% \\?(\w*)\*?:? (.*)$/
% Replace: 
% /\\discussion{\\\1** \2}/
