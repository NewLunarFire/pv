\begin{longtable}{l l X}
\point{Ouverture de l’assemblée}
\procedure{\textalpha\ ouvre l'assemblée à 17h38.}
\point{Nomination du présidium d’assemblée}
\proposition{\textbeta\ propose \textgamma\ et \textdelta\ au pr{\ae}sidium d'assemblée.}{\textepsilon}
\resultat{\pau}

\souspoint{Lecture et adoption de l’ordre du jour}
\proposition{\textzeta\ propose l'ordre du jour tel que présenté.}{\texteta}
\resultat{\pau}

\point{Ça commence}

\souspoint{On déboule}

\point{Divers}

\point{Levée de l’assemblée}
\procedure{Constatant l'épuisement de l'ordre du jour, \texttheta\ lève l'assemblée à 19h44.}
\end{longtable}